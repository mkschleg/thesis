%%%
% Text
%%%
\usepackage[utf8]{inputenc}
\usepackage[T1]{fontenc}    % use 8-bit T1 fonts
\usepackage{relsize}
\usepackage{microtype}      % microtypography

% Unicode characters
\DeclareUnicodeCharacter{1EF3}{\`y}

%%%
% Organization
%%%
\usepackage{subfiles}
\usepackage[titletoc]{appendix}
% For cross-references between files
% https://tex.stackexchange.com/questions/77774/undefined-control-sequence-when-cross-referencing-with-xr-hyper
\usepackage{nameref,zref-xr}
\zxrsetup{toltxlabel}

\usepackage[breaklinks=true,hidelinks]{hyperref} % See % https://en.wikibooks.org/wiki/LaTeX/Hyperlinks#Customization
\usepackage{url}

\usepackage{xspace}
\usepackage{xcolor}
\usepackage{csquotes} % Also used with biblatex
\usepackage{xmpincl} % Seems to be needed when converting to PDF/A, 

%%%
% For figures
%%%
\usepackage{graphicx} % more modern
\usepackage{subcaption}
\usepackage{sidecap}
\usepackage{tabularx}
\usepackage{pbox}
\usepackage{csquotes}
\usepackage{booktabs}
\usepackage{xcolor}
\usepackage{wrapfig}

\graphicspath{{./figures/}}

%%%
% Maths
%%%
% \usepackage{amssymb,amsmath}
% \usepackage{amsthm}
% \usepackage{amsfonts}       % blackboard math symbols
% \usepackage{nicefrac}       % compact symbols for 1/2, etc.
% \usepackage{mathtools}
% \usepackage{nicematrix}

% \usepackage{tabulary} % Better text wrapping in tables. See https://en.wikibooks.org/wiki/LaTeX/Tables
% \usepackage{rotating} % For the 'sidewaystable' environment. See https://en.wikibooks.org/wiki/LaTeX/Rotations
% \usepackage{multirow} % For multirow/multicolumn cells in a table. See https://en.wikibooks.org/wiki/LaTeX/Tables#Columns_spanning_multiple_rows


%\usepackage{subcaption} % A package that can be used to create sub-figures
\usepackage{enumitem} % for more control over enumerate
\usepackage{mdwlist}	% the 'note' environment

%%%%%%%%%%%%%%%
% biblatex    %
%%%%%%%%%%%%%%%
% (Added by Bernard Llanos)
% biblatex is intended to be the successor to BibTeX
% (https://en.wikibooks.org/wiki/LaTeX/Bibliography_Management#biblatex)
% \usepackage{cite}
\usepackage[%
backend=biber,%
style=numeric,%
backref=false,%
sortcites=true,sorting=nyt,%
mincitenames=1,maxcitenames=2%
]{biblatex}
% `backref=true` adds back references - Links to the in-text citations from
% the corresponding bibliography entries. Back references are not mentioned
% in thesis guidelines, but are, in my opinion, helpful for reading and editing.

\newcommand{\citep}{\parencite}
\newcommand{\citeA}[1]{\citeauthor{#1} (\citeyear{#1})}
\newcommand{\citeR}{\cite}


\usepackage[american]{babel}

% The following macro will put back references on the right edge of the page
% (https://tex.stackexchange.com/questions/149009/biblatex-pagebackref-reference-in-the-flush-right-margin)
\renewbibmacro*{pageref}{%
   \iflistundef{pageref}
     {\renewcommand{\finentrypunct}{\addperiod}}
     {\renewcommand{\finentrypunct}{\addspace}%
       \printtext{\addperiod\hfill\rlap{\hskip15pt\colorbox{blue!5}{\scriptsize\printlist[pageref][-\value{listtotal}]{pageref}}}}}}

\addbibresource{rnns.bib}
\addbibresource{rl.bib}
\addbibresource{candidacy.bib}
\addbibresource{predreps.bib}
\addbibresource{me.bib}

%%%%%%%%%%%%%%%%%%%%%%%%%%%%%%%%%%%%
% Glossary and acronyms (optional) %
%%%%%%%%%%%%%%%%%%%%%%%%%%%%%%%%%%%%
\usepackage[nogroupskip,nonumberlist,nopostdot]{glossaries}

\setacronymstyle{long-short-desc}
\setglossarystyle{altlist}

% A simple, but limited way to produce a sorted glossary
% Other options are described in the 'glossaries' Beginner's Guide (https://ctan.org/pkg/glossaries)
\makenoidxglossaries % use TeX to sort

% Glossary entries will be loaded from a separate file
\loadglsentries{\main/tex/glossary}

%%%%%%%%%%%%%%%%%%%%%%%%
% Thesis style package %
%%%%%%%%%%%%%%%%%%%%%%%%
\usepackage{\main/uathesis}  % Earlier version says this should be last package 
                     % imported. Never checked if this is still true. 
                     % Having this second last before the next one seems fine.

%%%%%%%%%%%%%%%%%%%%%%%%%%%%
% Shorthands               %
%%%%%%%%%%%%%%%%%%%%%%%%%%%%

% From the CVPR paper template (http://cvpr2017.thecvf.com/submission/main_conference/author_guidelines)
% Add a period to the end of an abbreviation unless there's one
% already, then \xspace.
\makeatletter
\DeclareRobustCommand\onedot{\futurelet\@let@token\@onedot}
\def\@onedot{\ifx\@let@token.\else.\null\fi\xspace}

\def\eg{\emph{e.g}\onedot} \def\Eg{\emph{E.g}\onedot}
\def\ie{\emph{i.e}\onedot} \def\Ie{\emph{I.e}\onedot}
\def\cf{\emph{c.f}\onedot} \def\Cf{\emph{C.f}\onedot}
\def\etc{\emph{etc}\onedot} \def\vs{vs\onedot}
\def\wrt{w.r.t\onedot} \def\dof{d.o.f\onedot}
\def\etal{\emph{et al}\onedot}
\makeatother

%%%%%%%%%%%%%%%%%%%%%%%%%%%%%%%%%%%%%%%%%%%
% Title page and Table of Contents Tweaks %
%%%%%%%%%%%%%%%%%%%%%%%%%%%%%%%%%%%%%%%%%%%

%% Correct title for TOC
\renewcommand{\contentsname}{Table of Contents}

% Fill in the following
% \title{Developing algorithms for learning predictive representations of state} % Title can't use formulae, symbols, superscripts, subscripts, greek letters, etc. all of which should be replaced with word substitutes
% \author{Matthew Schlegel}

\degree{\PhD}
%\degree{\PhD} % uncomment respective degree

\dept{Computing Science}  % Write Computing Science or Civil Engineering.

% If you have a specialization, uncomment the following line and enter it below.
% It must correspond with what it says on Bear Tracks
% (Academics>My Academics>Graduation). If, like most people, you don't have one,
% just leave it commented.
%\field{Specialization Field}

% Put the year that you submitted your thesis below
\submissionyear{\number2021}

%%%%%%%%%%%%%%%%%%%%%
% Document Content  %
%%%%%%%%%%%%%%%%%%%%%

% This is a modular document.
% The 'subfiles' package allows you to typeset the included
% documents separately from the main document, so that you
% can view only pieces of the thesis at a time.
% See https://en.wikibooks.org/wiki/LaTeX/Modular_Documents
%
% Subfiles that contain references: You can just run
% `biber subfilename` on them when compiling them individually.
% There is no need to make them reference the bibliography database
% 'refs.bib', as they inherit the reference from this file.


% \usepackage[showframe,headsep=1cm,headheight=2cm]{geometry}

% \usepackage{\main/uathesis}

\newcommand{\onlyinsubfile}[1]{#1}
\newcommand{\notinsubfile}[1]{}

% 
%%%
% Math packages
%%%
\usepackage{amssymb,amsmath}
\usepackage{amsthm}
\usepackage{amsfonts}       % blackboard math symbols
\usepackage{nicefrac}       % compact symbols for 1/2, etc.
\usepackage{mathtools}
\usepackage{nicematrix}


%%%%%%
% Theorems
%%%%%%
\newtheorem{theorem}{Theorem}[section]
\newtheorem{corollary}{Corollary}[theorem]
\newtheorem{lemma}[theorem]{Lemma}
\newtheorem{proposition}[theorem]{Proposition}
\newtheorem{assumption}{Assumption}


%%%%%
% Number sets
%%%%%
\newcommand{\Integers}{\mathbb{Z}}
\newcommand{\Naturals}{\mathbb{N}}
\newcommand{\Reals}{\mathbb{R}}
\newcommand{\RR}{\mathbb{R}}

%%%%%%
% Mathematical operations
%%%%%%

% Probabilites
\newcommand{\Prob}{\mathbb{P}}
\newcommand{\prob}{\mathbb{P}}

% Expectations
\newcommand{\E}{\mathbb{E}}
\newcommand{\Expected}{\mathbb{E}}
\newcommand{\Expectation}{\mathbb{E}}
% \newcommand{\expect}{\operatorname{\mathbb{E}}\expectarg}
% \DeclarePairedDelimiterX{\expectarg}[1]{[}{]}{%
% \ifnum\currentgrouptype=16 \else\begingroup\fi
% \activatebar#1
% \ifnum\currentgrouptype=16 \else\endgroup\fi
% }

% Variance
\newcommand{\V}{\mathbb{V}}
\newcommand{\Variance}{\mathbb{V}}

% define equals
\newcommand*{\defeq}{\stackrel{\mathsmaller{\mathsf{def}}}{=}}


% misc linear algebra
\newcommand{\trans}{\top}
\newcommand{\transpose}{{\raisebox{.2ex}{$\scriptscriptstyle\top$}}}
\newcommand{\inv}{{\raisebox{.2ex}{$\scriptscriptstyle-1$}}}
\newcommand{\invt}{{\raisebox{.2ex}{$\scriptscriptstyle-\top$}}}
\newcommand{\pinv}{{\raisebox{.2ex}{$\scriptscriptstyle\dagger$}}}
\newcommand{\pinvt}{{\raisebox{.2ex}{$\scriptscriptstyle\dagger\top$}}}

% vector w/ two things in it.
\newcommand{\twovec}[2]{{\small \Big[\!\! \begin{array}{c}
#1\\
#2 
\end{array}
\!\!\Big] }}

% Partial Derivative
\newcommand{\partialderivative}[2]{\frac{\partial #1}{\partial #2}}
\newcommand{\pd}[2]{\frac{\partial #1}{\partial #2}}

% Kronecker delta
\newcommand{\kron}{\delta}


%%%%%
% Variables
%%%%%

% Matrices
\newcommand{\Amat}{\mathbf{A}}
\newcommand{\Bmat}{\mathbf{B}}
\newcommand{\Cmat}{\mathbf{C}}
\newcommand{\Dmat}{\mathbf{D}}
\newcommand{\Hmat}{\mathbf{H}}
\newcommand{\Kmat}{\mathbf{K}}
\newcommand{\Lmat}{\mathbf{L}}
\newcommand{\Pmat}{\mathbf{P}}
\newcommand{\Ppi}{\mathbf{P}^\pi}
\newcommand{\Qmat}{\mathbf{Q}}
\newcommand{\Rmat}{\mathbf{R}}
\newcommand{\Tmat}{\mathbf{T}}
\newcommand{\Umat}{\mathbf{U}}
\newcommand{\Vmat}{\mathbf{V}}
\newcommand{\Wmat}{\mathbf{W}}
\newcommand{\Xmat}{\mathbf{X}}
\newcommand{\Zmat}{\mathbf{Z}}

\newcommand{\LL}{\lambda}

% Vectors
\newcommand{\avec}{\mathbf{a}}
\newcommand{\bvec}{\mathbf{b}}
\newcommand{\cvec}{\mathbf{c}}
\newcommand{\dvec}{\mathbf{d}}
\newcommand{\evec}{\mathbf{e}}
\newcommand{\gvec}{\mathbf{g}}
\newcommand{\hvec}{\mathbf{h}}
\newcommand{\lvec}{\mathbf{l}}
\newcommand{\mvec}{\mathbf{m}}
\newcommand{\nvec}{\mathbf{n}}
\newcommand{\pvec}{\mathbf{p}}
\newcommand{\qvec}{\mathbf{q}}
\newcommand{\rvec}{\mathbf{r}}
\newcommand{\svec}{\mathbf{s}}
\newcommand{\uvec}{\mathbf{u}}
\newcommand{\vvec}{\mathbf{v}}
\newcommand{\wvec}{\mathbf{w}}
\newcommand{\xvec}{\mathbf{x}}
\newcommand{\yvec}{\mathbf{y}}
\newcommand{\zvec}{\mathbf{z}}

% zero vector
\newcommand{\zerovec}{\mathbf{0}}

% yhat
\newcommand{\yhat}{\hat{y}}

%%%%
% Dimensions and sizes of things
%%%%
\newcommand{\xdim}{d}
\newcommand{\pdim}{k}
\newcommand{\nsamples}{T}
\newcommand{\nstates}{n}
\newcommand{\statesize}{n}
\newcommand{\obssize}{m}
\newcommand{\astatesize}{b}
\newcommand{\agentstatesize}{n}
\newcommand{\numgvfs}{N}
\newcommand{\numparams}{d}

%%%%
% various function parameters
%%%%
\newcommand{\regwgt}{\eta}
\newcommand{\decay}{\beta}
\newcommand{\stepsize}{\alpha}
\newcommand{\avewgt}{\epsilon}

%%%% 
% Learnable Weights
%%%%
\newcommand{\weightmat}{\Wmat}
\newcommand{\weightvec}{\wvec}
\newcommand{\weights}{\weightmat}
\newcommand{\weightspace}{\boldsymbol{\Omega}}
% \newcommand{\secweights}{\wvec}

%%%%
% Reinforcement learning specific
%%%%

% actions
\newcommand{\Actions}{\mathcal{A}}
\newcommand{\Action}{A}
\newcommand{\action}{a}

% environment states
\newcommand{\EnvStates}{\boldsymbol{\Psi}}
\newcommand{\EnvState}{S}
\newcommand{\envstate}{\psi}

% agent states
\newcommand{\States}{\mathcal{S}}
\newcommand{\State}{S}
\newcommand{\state}{s}
\newcommand{\pstate}{\tilde{s}}
\newcommand{\ipstate}{\tilde{s}}

% observations
\newcommand{\Observations}{\mathcal{O}}
\newcommand{\obs}{\mathbf{o}}

% features
\newcommand{\feat}{\mathbf{x}}

% histories
\newcommand{\Histories}{\mathcal{H}}
\newcommand{\Hists}{\mathcal{H}}
\newcommand{\Hist}{\mathcal{H}}
\newcommand{\history}{\hvec}

% agent state for po
\newcommand{\agentstate}{\svec}
\newcommand{\astate}{\svec}

% rewards and cumulants
\newcommand{\Rewards}{\mathcal{R}}
\newcommand{\reward}{r}
\newcommand{\cumulant}{c}

% MDP dynamic functions
\newcommand{\Pfcn}{\mathrm{P}}
\newcommand{\PSSA}{P_{s,s'}^a}
\newcommand{\Rfcn}{r}
\newcommand{\RSSA}{R_{s,s'}^a}

% Value functions
\newcommand{\Value}{\mathcal{V}}
\newcommand{\vfunc}{V}
\newcommand{\vifunc}[1]{V^{(#1)}}
\newcommand{\viweights}[1]{V^{(#1)}_\weights}

\newcommand{\QValue}{\mathcal{Q}}

\newcommand{\valuevec}{\mathbf{g}}
\newcommand{\rewardvec}{\mathbf{r}}

% RL policy induced distributions
\newcommand{\dmu}{\mathbf{d}_\mu}
\newcommand{\bpolicy}{\mu}
\newcommand{\tpolicy}{\pi}
\newcommand{\bpi}{\mathbf{b}}
\newcommand{\dstat}[1]{\mathbf{d}_{#1}}
\newcommand{\behaviorPolicy}{\mu}
\newcommand{\questionPolicy}{\pi}

% importance sampling
\newcommand{\rhomat}{\mathcal{P}}
\newcommand{\rhovec}{\boldsymbol{\rho}}

% error funcs
\newcommand{\tderror}{\boldsymbol{\delta}}
\newcommand{\tdnerror}{\boldsymbol{\delta_{TDN}}}
\newcommand{\loss}{\mathcal{L}}

% projection operator
\newcommand{\proj}{\Pi}


% Math Functions
\DeclareMathOperator*{\argmin}{argmin}
\DeclareMathOperator*{\diag}{diag}

% useful constants
\newcommand{\ralignspace}{\vspace{-0.15cm}}
\newcommand{\balignspace}{\par\vspace{-0.2cm}}



%%%%%%%%%%
%
% Paper Specific variables
% 
%%%%%%%%%%


%%%%
% Resampling
%%%%

\newcommand{\cumul}{c}
\newcommand{\cumulr}{C}
\newcommand{\stater}{S}
\newcommand{\actionr}{A}
% \renewcommand{\Value}{V}

\newcommand{\xiwer}{X_{\mathrm{IR}}}
\newcommand{\xbciwer}{X_{\mathrm{BC}}}
\newcommand{\xbciwerb}[1]{X_{\mathrm{BC}}^{(#1)}}
\newcommand{\xwis}{X_{\mathrm{WIS}^*}}
\newcommand{\xis}{X_{\mathrm{IS}}}
\newcommand{\xwindow}[1]{X_{#1}}
\newcommand{\Var}{\mathrm{Var}}

\newcommand{\bsize}{n}

\newcommand{\varReduction}{\V_{\text{reduc}}}
\newcommand{\irsample}{{i_j}}
\newcommand{\unisample}{{z_j}}
\newcommand{\muB}[1]{\mu_{#1}}

%%%%
% ARNNs
%%%%
% number of factors
\newcommand{\factors}{M}




% \newcommand{\dummycite}[1]{
%   {[{\bf CITE #1}]}}





\newcommand{\mytodo}[1]{{\color{red} #1}}
\newcommand{\citehere}{{\bf \color{red} [CITE]}}

%%% Local Variables:
%%% mode: latex
%%% TeX-master: "main.org"
%%% End:
