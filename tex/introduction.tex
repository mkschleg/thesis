% A workaround to allow relative paths in included subfiles
% that are to be compiled separately
% See https://tex.stackexchange.com/questions/153312/subfiles-inside-a-subfile-using-relative-paths
\providecommand{\main}{..}
\documentclass[\main/thesis.tex]{subfiles}

\begin{document}

\chapter{Introduction}

Here is a test reference~\cite{Knuth68:art_of_programming}.
These additional lines have been added just to demonstrate the spacing
for the rest of the document. Spacing will differ between the typeset main
document, and typeset individual documents, as the commands
to change spacing for the body of the thesis are only in the main document.

\section{Cross-Referencing}\label{sec:crossRef}

Cross-references between child documents are possible using the
\href{https://www.ctan.org/pkg/zref}{\texttt{zref}} package.

\newpage

Text on a new page, to test top margin size.

A sample equation \eqref{eq:test} follows:

\begin{equation}
y = \frac{1}{x^2} + 4 \label{eq:test}
\end{equation}

A sample table, Table \ref{tab:test}:

\begin{table}[h]
    \centering
    \begin{tabulary}{0.75\textwidth}{r|L}
    \textbf{Non-wrapping column} & \textbf{Wrapping column} \\ \hline
    This is an ordinary column & This is a balanced-width column, where text will wrap
    \end{tabulary}
    \caption[A sample table] {A sample table created using the \href{https://ctan.org/pkg/tabulary}{\texttt{tabulary}} package}
    \label{tab:test}
\end{table}

If there are many acronyms, such as \gls{asa}, and specialized technical terms, consider adding a glossary.
Sample \gls{sampleGlossaryEntry}, and acroynm (\gls{asa}) descriptions are provided above.

\end{document}